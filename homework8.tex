\documentclass[fleqn]{article}
\oddsidemargin 0.0in
\textwidth 6.0in
\thispagestyle{empty}
\usepackage{import}
\usepackage{amsmath}
\usepackage{graphicx}
\usepackage{flexisym}
\usepackage{calligra}
\usepackage{amssymb}
\usepackage{bigints} 
\usepackage[english]{babel}
\usepackage[utf8x]{inputenc}
\usepackage{float}
\usepackage[colorinlistoftodos]{todonotes}


\DeclareMathAlphabet{\mathcalligra}{T1}{calligra}{m}{n}
\DeclareFontShape{T1}{calligra}{m}{n}{<->s*[2.2]callig15}{}
\newcommand{\scriptr}{\mathcalligra{r}\,}
\newcommand{\boldscriptr}{\pmb{\mathcalligra{r}}\,}

\definecolor{hwColor}{HTML}{1a0252}

\begin{document}

  \begin{titlepage}

    \newcommand{\HRule}{\rule{\linewidth}{0.5mm}}

    \center

    \begin{center}
      \includegraphics[height=11cm, width=11cm]{asu.png}
    \end{center}

    \vline

    \textsc{\LARGE Quantum Physics II}\\[1.5cm]

    \HRule \\[0.5cm]
    { \huge \bfseries Problem Set Eight}\\[0.4cm] 
    \HRule \\[1.0cm]

    \textbf{Behnam Amiri}

    \bigbreak

    \textbf{Prof: Onur Erten}

    \bigbreak

    \textbf{{\large \today}\\[2cm]}

    \vfill

  \end{titlepage}

  \begin{enumerate}
    \item A particle of spin $\dfrac{1}{2}$ is in a $p$ orbital $(L=1)$ of some potential.
    \begin{enumerate}
      \item What values of $J$, total angular momentum can it have?

        % \textcolor{hwColor}{
        %   \\
        % }

      \item Use the method described in lecture (using the z-components and $\pm$ components of $J=L+S$
      to find the state $|j=1/2, j_z=1/2 \rangle$ in terms of the basis $|\ell=1, \ell_z \rangle|s=1/2, s_z \rangle$)
      Hint: First work out the $|j=3/2, j_z=3/2 \rangle$ then use orthogonality principle).

        % \textcolor{hwColor}{
        %   \\
        % }

    \end{enumerate}

    \item \textbf{8.4, Zettili} Two identical particles of spin $\dfrac{1}{2}$ are enclosed in a one-dimensional 
    box potential of length $L$ with rigid walls at $x=0$ and $x=L$. Assuming that the two-particle system
    is in a triplet spin state, find the energy levels, the wave functions, and the degeneracies corresponding 
    to the three lowest states.

      % \textcolor{hwColor}{
      %   \\
      % }

  \end{enumerate}

\end{document}
