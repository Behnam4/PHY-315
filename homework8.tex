\documentclass[fleqn]{article}
\oddsidemargin 0.0in
\textwidth 6.0in
\thispagestyle{empty}
\usepackage{import}
\usepackage{amsmath}
\usepackage{graphicx}
\usepackage{flexisym}
\usepackage{calligra}
\usepackage{amssymb}
\usepackage{bigints} 
\usepackage[english]{babel}
\usepackage[utf8x]{inputenc}
\usepackage{float}
\usepackage[colorinlistoftodos]{todonotes}


\DeclareMathAlphabet{\mathcalligra}{T1}{calligra}{m}{n}
\DeclareFontShape{T1}{calligra}{m}{n}{<->s*[2.2]callig15}{}
\newcommand{\scriptr}{\mathcalligra{r}\,}
\newcommand{\boldscriptr}{\pmb{\mathcalligra{r}}\,}

\definecolor{hwColor}{HTML}{1a0252}

\begin{document}

  \begin{titlepage}

    \newcommand{\HRule}{\rule{\linewidth}{0.5mm}}

    \center

    \begin{center}
      \includegraphics[height=11cm, width=11cm]{asu.png}
    \end{center}

    \vline

    \textsc{\LARGE Quantum Physics II}\\[1.5cm]

    \HRule \\[0.5cm]
    { \huge \bfseries Problem Set Eight}\\[0.4cm] 
    \HRule \\[1.0cm]

    \textbf{Behnam Amiri}

    \bigbreak

    \textbf{Prof: Onur Erten}

    \bigbreak

    \textbf{{\large \today}\\[2cm]}

    \vfill

  \end{titlepage}

  \begin{enumerate}
    \item A particle of spin $\dfrac{1}{2}$ is in a $p$ orbital $(\ell=1)$ of some potential.
    \begin{enumerate}
      \item What values of $J$, total angular momentum can it have?

        \textcolor{hwColor}{
          \\
          Knowing that the particle has $s=\dfrac{1}{2}, \ell=1$ and considering that $J=L+S$, by the 
          triangle rule gives $J=\left\{\dfrac{1}{2}, \dfrac{3}{2}\right\}$.
          \\
        }

      \item Use the method described in lecture (using the z-components and $\pm$ components of $J=L+S$
      to find the state $|j=1/2, j_z=1/2 \rangle$ in terms of the basis $|\ell=1, \ell_z \rangle|s=1/2, s_z \rangle$)
      Hint: First work out the $|j=3/2, j_z=3/2 \rangle$ then use orthogonality principle).

        \textcolor{hwColor}{
          \\
          $j=3/2$ and $j_z=3/2$ can be reached in one manner.
          \\
          \\
          $
            |j=3/2, j_z=3/2 \rangle=| \ell=1, \ell_z=1 \rangle ~ | s=1/2, s_z=1/2 \rangle
            \\
            \\
            \\
            J-|j=\dfrac{3}{2}, j_z=\dfrac{3}{2}|=\hbar \sqrt{J\bigg( j+1 \bigg)-j_z \bigg( j_z-1\bigg) } ~ | j, j_z-1 \rangle
            =\hbar ~ \sqrt{3} | j=3/2, j_z=1/2 \rangle
            \\
            \\
            \\
            \bigg( L_- + S_-\bigg) ~ | j=3/2, j_z=3/2 \rangle= \bigg( L_-+S_-\bigg) ~ | \ell=1, \ell_z=1 \rangle ~ | S=1/2, S_z=1/2 \rangle
            \\
            \\
            \\
            =\bigg( L_- |\ell=1, \ell_z=1 \rangle \bigg) ~ \bigg( S=1/2, S_z=1/2 \bigg)
            +|\ell=1, \ell_z=1 \rangle ~ \bigg( S_- | S=1/2, S_z=1/2 \rangle \bigg)
            \\
            \\
            \\
            =\hbar \sqrt{1+(1+1)-(1-1)} ~ |\ell=1, \ell_z=0 \rangle ~ | S=1/2, S_z=1/2 \rangle
            +| \ell=1 \rangle ~ | \ell_z=1 \rangle 
            \\
            \\
            \bigg( 
              \hbar \sqrt{1/2+(1/2+1)-1/2 (1/2-1)} ~ |S=1/2, S_z=1/2 \rangle  
            \bigg)
            \\
            \\
            \\
            \\
            \therefore ~~~ \sqrt{3} ~ \hbar ~ |\dfrac{3}{2}, \dfrac{1}{2} \rangle
            =\hbar ~ |1,1 \rangle | \dfrac{1}{2}, -\dfrac{1}{2} \rangle+\sqrt{2} \hbar |1,0 \rangle ~ |\dfrac{1}{2}, \dfrac{1}{2} \rangle
            \\
            \\
            \\
          $
          What we are looking for is $|\dfrac{1}{2}, \dfrac{1}{2} \rangle$. Posibilities with $j_z=\dfrac{1}{2}$ is that 
          $| \ell=1, \ell_z \rangle ~ |S=1/2, S_z$ basis are in $| \dfrac{3}{2}, \dfrac{1}{2} \rangle$ which are;
          \\
          \\
          $
            \begin{cases}
              |\ell=1, \ell_z=1 \rangle ~ |S=1/2, S_z=-1/2 \rangle
              \\
              \\
              |\ell=1, \ell_z=0 \rangle ~ |S=1/2, S_z=1/3 \rangle
            \end{cases}
            \\
            \\
          $
          $|j=1/2, j_z=1/2 \rangle$ is orthogonal and normalized to $|j=3/2, j_z=1/2 \rangle$ then we have:
          \\
          \\
          $
           \therefore ~~~ \boxed{
            |j=1/2, j_z=1/2 \rangle=\sqrt{\dfrac{3}{2}} ~ |\ell=1, \ell_z=1 \rangle ~ |S=1/2, S_z=-1/2 \rangle
            -\dfrac{1}{\sqrt{3}} ~ |\ell=1, \ell_z=0 \rangle ~ |S=1/2, S_z=1/2 \rangle
           } ~~~~ \checkmark
            \\
            \\
          $ 
        }

      \item If the system is initially prepared in the state $|j=1/2, j_z=1/2 \rangle$, what is the 
      probability that measuring its value of $s_z$ would give $+1/2$?

        \textcolor{hwColor}{
          \\
          In the previous section we found for the state $|j=1/2, j_z=1/2 \rangle$ the basis. Therfore,
          the probability of measuring its value of $s_z$ gives $|-\dfrac{1}{\sqrt{3}}|^2=\dfrac{1}{3} ~~~~ \checkmark$.
          \\
          \\
        }

    \end{enumerate}

    \item \textbf{8.4, Zettili} Two identical particles of spin $\dfrac{1}{2}$ are enclosed in a one-dimensional 
    box potential of length $L$ with rigid walls at $x=0$ and $x=L$. Assuming that the two-particle system
    is in a triplet spin state, find the energy levels, the wave functions, and the degeneracies corresponding 
    to the three lowest states.

      \textcolor{hwColor}{
        \\
        The spin triplet combination with two $S=\dfrac{1}{2}$ particles has a spin move function given by any of the 
        symmetric S states.
        \\
        \\
        $
          \chi_{s}=\begin{cases}
            \uparrow \uparrow = |1,1 \rangle=|1/2, 1/2 \rangle ~ |1/2, 1/2 \rangle
            \\
            \\
            \uparrow \downarrow +\downarrow \uparrow=|1, 0 \rangle=\dfrac{1}{\sqrt{2}} \bigg( 
              |1/2, 1/2 \rangle ~ |1/2, -1/2 \rangle+|1/2, -1/2 \rangle ~ |1/2, 1/2 \rangle  
            \bigg)
            \\
            \\
            \downarrow \downarrow=|1, -1 \rangle=|1/2, -1/2 \rangle ~ |1/2, -1/2 \rangle
          \end{cases}
          \\
          \\
        $
        The eigenstate $|n \rangle$ with $\Psi_n\bigg( x \bigg)=\langle x | n \rangle=\sqrt{\dfrac{2}{L}} ~ sin\bigg( \dfrac{n \pi x}{L}\bigg)$
        and the energy $E_n=\pi^2 \hbar^2 n^2/2,L^2$ and the antisymmetric combination is
        $\dfrac{1}{\sqrt{2}} \left[|m \rangle ~ |n \rangle -|n \rangle ~ | m \rangle\right]$  (nonzero when $m \neq n$).
        \\
        \\
        $
          \Psi_{mn}=\dfrac{2}{L} \left[
            sin \bigg( \dfrac{m \pi x_1}{L} \bigg) ~ sin\bigg( \dfrac{n \pi x_2}{L} \bigg)-sin \bigg( \dfrac{n \pi x_1}{L} \bigg) ~ sin \bigg( \dfrac{m \pi x_2}{L} \bigg) \chi_5
          \right]
          \\
          \\
          \\
          E=E_m+E_n=\dfrac{\pi \hbar^2}{2 m L^2} \bigg( m^2+n^2\bigg)
          \\
          \\
        $
        The three loweest states are calculated from $(m,n)$ when they are not equal integers. 
        \\
        \\
        $
          \bigg( 1^2+2^2\bigg) < 1^2+3^2 < 2^2+3^2 < 1^2+4^2
          \\
          \\
          \text{The grand state has } (1,2) ~ or ~ (2,1)
          \\
          \\
          \therefore ~~~ \Psi_{12}=\dfrac{2}{L} \left[
            sin \bigg( \dfrac{\pi x_1}{L} \bigg) ~ sin\bigg( \dfrac{2 \pi x_2}{L} \bigg)-sin \bigg( \dfrac{2 \pi x_1}{L} \bigg) ~ sin \bigg( \dfrac{\pi x_2}{L} \bigg) \chi_5
          \right]
          \\
          \\
          \\
          E_{12}=\dfrac{5 \pi^2 \hbar^2}{2mL^2}=5 ~ E_0
          \\
          \\
          \\
          \text{First excited state}
          \\
          \\
          \Psi_{13}=\dfrac{2}{L} \left[
            sin \bigg( \dfrac{\pi x_1}{L} \bigg) ~ sin\bigg( \dfrac{3 \pi x_2}{L} \bigg)-sin \bigg( \dfrac{3 \pi x_1}{L} \bigg) ~ sin \bigg( \dfrac{\pi x_2}{L} \bigg)
          \right]
          \\
          \\
          E_{13}=10 E_0
          \\
          \\
          \\
          \text{Second excited state}
          \\
          \\
          \Psi_{23}=\dfrac{2}{L} \left[
            sin \bigg( \dfrac{ 2 \pi x_1}{L} \bigg) ~ sin\bigg(\dfrac{3 \pi x_2}{L} \bigg)-sin \bigg( \dfrac{3 \pi x_1}{L} \bigg) ~ sin \bigg( \dfrac{2 \pi x_2}{L} \bigg)
          \right]
          \\
          \\
          E_{23}=13 E_0
          \\
          \\
        $
      }

  \end{enumerate}

\end{document}
