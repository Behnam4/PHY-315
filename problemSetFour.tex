\documentclass[fleqn]{article}
\oddsidemargin 0.0in
\textwidth 6.0in
\thispagestyle{empty}
\usepackage{import}
\usepackage{amsmath}
\usepackage{graphicx}
\usepackage{flexisym}
\usepackage{calligra}
\usepackage{amssymb}
\usepackage{bigints} 
\usepackage[english]{babel}
\usepackage[utf8x]{inputenc}
\usepackage{float}
\usepackage[colorinlistoftodos]{todonotes}


\DeclareMathAlphabet{\mathcalligra}{T1}{calligra}{m}{n}
\DeclareFontShape{T1}{calligra}{m}{n}{<->s*[2.2]callig15}{}
\newcommand{\scriptr}{\mathcalligra{r}\,}
\newcommand{\boldscriptr}{\pmb{\mathcalligra{r}}\,}

\definecolor{hwColor}{HTML}{1a0252}

\begin{document}

  \begin{titlepage}

    \newcommand{\HRule}{\rule{\linewidth}{0.5mm}}

    \center

    \begin{center}
      \includegraphics[height=11cm, width=11cm]{asu.png}
    \end{center}

    \vline

    \textsc{\LARGE Quantum Physics II}\\[1.5cm]

    \HRule \\[0.5cm]
    { \huge \bfseries Problem Set 4}\\[0.4cm] 
    \HRule \\[1.0cm]

    \textbf{Behnam Amiri}

    \bigbreak

    \textbf{Prof: Onur Erten}

    \bigbreak

    \textbf{{\large \today}\\[2cm]}

    \vfill

  \end{titlepage}

  \begin{enumerate}
    \item \textbf{4-10} 
    \begin{enumerate}
      \item Check that $Arj_1(kr)$ satisfies the radial equation with $V(r)=0$ and $\ell=1$.
      \item Determine graphically the allowed energies for the infinite spherical well when $\ell=1$.
    \end{enumerate}

    \item \textbf{4-48} 
    \begin{enumerate}
      \item Prove the three-dimential virial theorem
      $$2\langle T \rangle=\langle r.\nabla V\rangle$$
      (for stationary states). Hint: refer to problem 3.37.

      \textcolor{hwColor}{
        \\
        \\
        Let's recall what the vir­ial the­o­rem is. The vir­ial the­o­rem re­lates the ex­pec­ta­tion ki­netic en­ergy of a quan­tum sys­tem to the po­ten­tial. 
        That is of the­o­ret­i­cal in­ter­est, as well as im­por­tant for com­pu­ta­tional meth­ods like \emph{den­sity func­tional the­ory}.
        Con­sider a quan­tum sys­tem in a state of def­i­nite en­ergy $E$. In other words, con­sider a quan­tum sys­tem in a sta­tion­ary state. 
        \\
        It does not have to be the ground state. The quan­tum sys­tem will be as­sumed to be in in­fi­nite space. To keep it sim­ple, 
        for now as­sume that there is a sin­gle par­ti­cle with po­si­tion vec­tor ${\skew0\vec r}$ in a po­ten­tial $V({\skew0\vec r})$. 
        That cov­ers our pre­vi­ous ex­am­ples of the har­monic os­cil­la­tor and the hy­dro­gen atom.
        Then the vir­ial the­o­rem re­lates the ex­pec­ta­tion ki­netic en­ergy $\left\langle{T}\right\rangle $ to the po­ten­tial $V$ as fol­lows:
        $$2\langle T \rangle=\langle r.\nabla V\rangle$$
        \\
        $
          \dfrac{d}{dt} \langle r.p \rangle=\dfrac{i}{\hbar} \langle \left[H, r.p\right] \rangle
          \\
          \\
          \left[H, r.p\right]
          =\sum\limits_{i=1}^{3} \left[H, r_i p_i\right]
          =\sum\limits_{i=1}^{3} \left(\left[H, p_i\right]r_i+\left[H, r_i\right]p_i\right)
          =\sum\limits_{i=1}^{3} \left(
            r_i \left[V, p_i\right]+p_i \dfrac{1}{2m} \left[p^2, r_i\right]
          \right)
          \\
          \\
          \\
          \left[p^2, r_i\right]
          =\sum\limits_{i=1}^{3} \left[p_j p_j, r_i\right]
          =\sum\limits_{i=1}^{3} \left[p_j \left(-i \delta_{ij}\right)+p_j \left(-i \hbar \delta_{ij}\right)\right]
          \\
          \\
          \\
          \therefore ~~~ \left[p^2, r_i\right]=-2i \hbar p_i ~~~~ \checkmark
        $
        \\
        \\
        Now we need to find $\left[V, p_i\right]$. Recall that
        \\
        \\
        $
          \left[f, p\right]g=f \dfrac{\hbar}{i} \dfrac{dg}{dx}-\dfrac{\hbar}{i} \dfrac{d}{dx} \left(fg\right)
          =f \dfrac{\hbar}{i} \dfrac{dg}{dx}-\dfrac{\hbar}{i} \left(g \dfrac{df}{dx}+f \dfrac{dg}{dx}\right)
          =i g \hbar \dfrac{df}{dx}
          \\
          \\
          \\
          \therefore ~~~ \left[f, p\right]g=i g \hbar \dfrac{df}{dx}
          \\
          \\
          \\
          \therefore ~~~ \begin{cases}
            \left[f, p\right]=i \hbar \dfrac{df}{dx} ~~~~ \checkmark
            \\
            \\
            \left[V, p_i\right]=i \hbar \dfrac{\partial V}{\partial r_i} ~~~~ \checkmark
          \end{cases}
          \\
          \\
          \\
          \left[H, r.p\right]=\sum\limits_{i=1}^{3} \left[
            \dfrac{-i \hbar}{m} p_i p_i +r_i \left(i \hbar \dfrac{\partial V}{\partial r_i}\right)
          \right]
          =i \hbar \left(-\dfrac{p^2}{m}+r. \nabla V\right)
          \\
          \\
          \\
          \dfrac{d}{dt} \langle r.p \rangle
          =\langle \dfrac{p^2}{m}-r. \nabla V \rangle
          =2 \langle T \rangle-\langle r. \nabla V \rangle
        $
        \\
        \\
        Since we are told that our case is for stationary states then $\dfrac{d}{dt} \langle r.p \rangle=0$. Hence,
        \\
        \\
        $
          2 \langle T \rangle-\langle r. \nabla V \rangle=0
          \\
          \\
          \\
          \therefore ~~~ 2 \langle T \rangle=\langle r. \nabla V \rangle ~~~~ \checkmark
        $
        \\
        \\
      }
      \item Apply the virial theorem to the case of hydrogen, and show that
      $$\langle T \rangle=-E_n; ~~~~~ \langle V \rangle=2E_n$$

      \textcolor{hwColor}{
        The hydrogen atom consists of a heavy, essentially motionless proton of charge $e$, with a much lighter electron that orbits around it.
        From Coulomb's law, the potential energy of electron (This is what goes into the Schr$\ddot{o}$dinger equation-not the electric potential
        $e/4 \pi \epsilon_0 r$) is
        $
          V(r)=-\dfrac{e^2}{4 \pi \epsilon_0} \dfrac{1}{r}
          \\
          \\
          \\
          \nabla V=\dfrac{\partial }{\partial r} \left(-\dfrac{e^2}{4 \pi \epsilon_0} \dfrac{1}{r}\right) \hat{r}
          =-\dfrac{e^2}{4 \pi \epsilon_0} \hat{r} \dfrac{d}{dr} \left(\dfrac{1}{r}\right)
          =-\dfrac{e^2}{4 \pi \epsilon_0} \dfrac{-1}{r^2} \hat{r}
          \\
          \\
          \\
          \therefore ~~~ \nabla V=\dfrac{e^2}{4 \pi \epsilon_0} \dfrac{1}{r^2} \hat{r} ~~~~ \checkmark
          \\
          \\
          \\
          r.\nabla V=\left(r \hat{r}\right).\left(\dfrac{e^2}{4 \pi \epsilon_0} \dfrac{1}{r^2} \hat{r}\right)
          =\dfrac{e^2}{4 \pi \epsilon_0} \dfrac{1}{r}
          \\
          \\
          \\
          \therefore ~~~ 2 \langle T \rangle=\langle r. \nabla V \rangle=\langle \dfrac{e^2}{4 \pi \epsilon_0} \dfrac{1}{r} \rangle=-\langle V \rangle
          \\
          \\
        $
        For all atoms and molecules, the motions of the planets $\langle T \rangle=-\dfrac{1}{2} \langle V \rangle$. But for a
        harmonic oscillator, $V=\dfrac{1}{2} kx^2$, $n=2$ so that $\langle T \rangle=\langle V \rangle$ again in either 
        classical or quantum mechanics. Therefore we have the following
        \\
        \\
        $
          \langle T \rangle=\langle V \rangle=E_n \Longrightarrow \langle T \rangle-2\langle T \rangle=E_n
          \\
          \\
          \\
          \therefore ~~~ \begin{cases}
            \langle T \rangle=-E_n
            \\
            \\
            \langle V \rangle=2 E_n
          \end{cases} ~~~~~~~ \checkmark
        $
      }

    \end{enumerate}
    
  \end{enumerate}

\end{document}