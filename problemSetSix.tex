\documentclass[fleqn]{article}
\oddsidemargin 0.0in
\textwidth 6.0in
\thispagestyle{empty}
\usepackage{import}
\usepackage{amsmath}
\usepackage{graphicx}
\usepackage{flexisym}
\usepackage{calligra}
\usepackage{amssymb}
\usepackage{bigints} 
\usepackage[english]{babel}
\usepackage[utf8x]{inputenc}
\usepackage{float}
\usepackage[colorinlistoftodos]{todonotes}


\DeclareMathAlphabet{\mathcalligra}{T1}{calligra}{m}{n}
\DeclareFontShape{T1}{calligra}{m}{n}{<->s*[2.2]callig15}{}
\newcommand{\scriptr}{\mathcalligra{r}\,}
\newcommand{\boldscriptr}{\pmb{\mathcalligra{r}}\,}

\definecolor{hwColor}{HTML}{1a0252}

\begin{document}

  \begin{titlepage}

    \newcommand{\HRule}{\rule{\linewidth}{0.5mm}}

    \center

    \begin{center}
      \includegraphics[height=11cm, width=11cm]{asu.png}
    \end{center}

    \vline

    \textsc{\LARGE Quantum Physics II}\\[1.5cm]

    \HRule \\[0.5cm]
    { \huge \bfseries Problem Set Six}\\[0.4cm] 
    \HRule \\[1.0cm]

    \textbf{Behnam Amiri}

    \bigbreak

    \textbf{Prof: Onur Erten}

    \bigbreak

    \textbf{{\large \today}\\[2cm]}

    \vfill

  \end{titlepage}

  \begin{enumerate}
    \item \textbf{4-24} 
    \begin{enumerate}
      \item Derive Equation $4.131$ from Equation $4.130$. \emph{Hint:} Use a test function; otherwise
      you are likely to drop some terms.

        \textcolor{hwColor}{
          \\
          From the textbook we have.
          \\
          \\
          $
            \begin{cases}
              L_{\pm}=\pm \hbar e^{\pm i \phi} \left(
                \dfrac{\partial}{\partial\theta}
                \pm ~ i ~ cot\theta \dfrac{\partial}{\partial\phi}
              \right) ~~~~~~~~~~~~~~~~~~~~~~~~~~~~~~~~~ (4.130)
              \\
              \\
              L_+ ~ L_-=-\hbar^2 \left(
                \dfrac{\partial^2}{\partial \theta^2}
                +cot\theta ~ \dfrac{\partial}{\partial \theta}
                +cot^2\theta \dfrac{\partial^2}{\partial \phi^2}
                +i ~ \dfrac{\partial}{\partial \phi}
              \right) ~~~~~~~ (4.131)
            \end{cases}
            \\
            \\
            \\
            L_+ ~ L_- \times f=-\hbar^2 ~ e^{i \phi} \left(\dfrac{\partial}{\partial \theta}+i ~ cot\theta \dfrac{\partial}{\partial \phi}\right)
            \left(
              e^{-i \phi} \dfrac{\partial f}{\partial \theta}-i e^{-i \phi} cot\theta \dfrac{\partial f}{\partial \phi}
            \right)
            \\
            \\
            \\
            =-\hbar^2 ~ e^{i \phi} (
              e^{-i \phi} \left[\dfrac{\partial^2 f}{\partial \theta^2}+ i ~ csc^2 \theta \dfrac{\partial f}{\partial \phi}-i ~ cot\theta \dfrac{\partial^2 f}{\partial\theta \partial \phi}\right]
              \\
              +i ~ cot\theta \left[-i ~ e^{-i\phi} \left(\dfrac{\partial f}{\partial \theta}-i~cot\theta \dfrac{\partial f}{\partial \phi}\right)+ e^{-i \phi} \left(\dfrac{\partial^2 f}{\partial \phi \partial \theta}-i ~ cot\theta \dfrac{\partial^2 f}{\partial \phi^2}\right)   \right]
            )
            \\
            \\
            \\
            =-\hbar^2 ~ \left[
              \dfrac{\partial^2 f}{\partial \theta^2}
              +cot\theta \dfrac{\partial f}{\partial \theta}
              +cot^2\theta \dfrac{\partial^2 f}{\partial \phi^2}
              +i \left(
                csc^2\theta \dfrac{\partial f}{\partial \phi}
                -cot\theta \dfrac{\partial^2 f}{\partial\theta \partial\phi}
                -cot^2\theta \dfrac{\partial f}{\partial \phi}
                +cot\theta \dfrac{\partial^2 f}{\partial\phi \partial\theta}
              \right)
            \right]
            \\
            \\
          $
          \\
          Factoring out $f$.
          \\
          \\
          $
            =-\hbar^2 ~ f\left[
              \dfrac{\partial^2}{\partial\theta^2}+cot\theta \dfrac{\partial}{\partial \theta}+cot^2\theta \dfrac{\partial^2}{\partial \phi^2}
              -i ~ cot^2 \theta \dfrac{\partial}{\partial \phi} 
              +i ~ csc^2\theta \dfrac{\partial}{\partial \phi}
            \right]
            \\
            \\
            \\
            \therefore ~~~ L_+ ~ L_-=-\hbar^2 \left(
              \dfrac{\partial^2}{\partial \theta^2}
              +cot\theta ~ \dfrac{\partial}{\partial \theta}
              +cot^2\theta \dfrac{\partial^2}{\partial \phi^2}
              +i ~ \dfrac{\partial}{\partial \phi}
            \right) ~~~~ \checkmark
          $
        }

      \pagebreak

      \item Derive Equations $4.132$ from Equations $4.129$ and $4.131$. \emph{Hint:} Use Equation $4.112$.

        \textcolor{hwColor}{
          \\
          From the textbook we have.
          \\
          \\
          $
            \begin{cases}
              L^2=L_{\pm} ~ L_{\mp}+L^2_z \mp \hbar L_z ~~~~~~~~~~~~~~~~~~~~~~~~~~~~~~~~~~~~~~~~~~~ (4.112)
              \\
              \\
              L_z=-i \hbar \dfrac{\partial}{\partial \phi} ~~~~~~~~~~~~~~~~~~~~~~~~~~~~~~~~~~~~~~~~~~~~~~~~~~~~~~~~~~~ (4.129)
              \\
              \\
              L_+ ~ L_-=-\hbar^2 \left(
                \dfrac{\partial^2}{\partial \theta^2}
                +cot\theta ~ \dfrac{\partial}{\partial \theta}
                +cot^2\theta \dfrac{\partial^2}{\partial \phi^2}
                +i ~ \dfrac{\partial}{\partial \phi}
              \right) ~~~~~~~ (4.131)
              \\
              \\
              L^2=-\hbar^2 \left[
                \dfrac{1}{sin\theta} \dfrac{\partial}{\partial \theta} \left(sin\theta \dfrac{\partial}{\partial \theta}\right)
                +\dfrac{1}{sin^2\theta} \dfrac{\partial^2}{\partial \phi^2}
              \right] ~~~~~~~~~~~~~~~~~ (4.132)
            \end{cases}
            \\
            \\
            \\
            \\
            L^2=-\hbar^2 \left(\dfrac{\partial^2}{\partial\theta^2}+cot\theta \dfrac{\partial}{\partial\theta}+cot^2\theta \dfrac{\partial^2}{\partial\phi^2}+i \dfrac{\partial}{\partial \phi}\right)
            -\hbar^2 \dfrac{\partial^2}{\partial \phi^2}
            -\hbar \dfrac{\partial}{\partial \phi} \dfrac{\hbar}{i}
            \\
            \\
            \\
            =\hbar^2 \left(\dfrac{\partial^2}{\partial\theta}+\dfrac{\partial^2}{\partial \phi^2}+cos^2\theta \dfrac{\partial^2}{\partial \phi^2}\right)
            \\
            \\
            \\
            =\hbar^2 \left(
              \dfrac{\partial^2}{\partial \theta^2}
              +cot\theta \dfrac{\partial}{\partial \theta}
              +\dfrac{1}{sin^2 \theta} \dfrac{\partial^2}{\partial \phi^2}
            \right)
            \\
            \\
            \\
            \\
            \therefore ~~~ L^2=-\hbar^2 \left[
              \dfrac{1}{sin\theta} \dfrac{\partial}{\partial \theta} \left(sin\theta \dfrac{\partial}{\partial \theta}\right)
              +\dfrac{1}{sin^2\theta} \dfrac{\partial^2}{\partial \phi^2}
            \right] ~~~~ \checkmark
          $
          \\
          \\
        }

    \end{enumerate}


    \item \textbf{5-11} Consider the wave function
    $$
      \psi(\theta, \phi)=3 ~ sin\theta ~ cos\theta ~ e^{i \phi}-2\left(1-cos^2\theta\right) e^{2i\phi}
    $$
    \begin{enumerate}
      \item Write $\psi(\theta, \phi)$ in terms of the spherical harmonics.

        % \textcolor{hwColor}{
        %   \\
        % }


      \item Write the expression found in $(a)$ in terms of the Cartesian coordinates.

        % \textcolor{hwColor}{
        %   \\
        % }

      \item Is $\psi(\theta, \phi)$ an eigenstate of $\hat{\overrightarrow{L}}^2$ or $\hat{L}_z$?

        % \textcolor{hwColor}{
        %   \\
        % }

      \item Find the probability of measuring $2 \hbar$ for the z-component of the orbital angular momentum.

        % \textcolor{hwColor}{
        %   \\
        % }

    \end{enumerate}

  \end{enumerate}

\end{document}
