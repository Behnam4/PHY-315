\documentclass[fleqn]{article}
\oddsidemargin 0.0in
\textwidth 6.0in
\thispagestyle{empty}
\usepackage{import}
\usepackage{amsmath}
\usepackage{graphicx}
\usepackage{flexisym}
\usepackage{calligra}
\usepackage{amssymb}
\usepackage{bigints} 
\usepackage[english]{babel}
\usepackage[utf8x]{inputenc}
\usepackage{float}
\usepackage[colorinlistoftodos]{todonotes}


\DeclareMathAlphabet{\mathcalligra}{T1}{calligra}{m}{n}
\DeclareFontShape{T1}{calligra}{m}{n}{<->s*[2.2]callig15}{}
\newcommand{\scriptr}{\mathcalligra{r}\,}
\newcommand{\boldscriptr}{\pmb{\mathcalligra{r}}\,}

\definecolor{hwColor}{HTML}{1a0252}

\begin{document}

  \begin{titlepage}

    \newcommand{\HRule}{\rule{\linewidth}{0.5mm}}

    \center

    \begin{center}
      \includegraphics[height=11cm, width=11cm]{asu.png}
    \end{center}

    \vline

    \textsc{\LARGE Quantum Physics II}\\[1.5cm]

    \HRule \\[0.5cm]
    { \huge \bfseries Problem Set Seven}\\[0.4cm] 
    \HRule \\[1.0cm]

    \textbf{Behnam Amiri}

    \bigbreak

    \textbf{Prof: Onur Erten}

    \bigbreak

    \textbf{{\large \today}\\[2cm]}

    \vfill

  \end{titlepage}

  \begin{enumerate}
    \item \textbf{4-29}
    \begin{enumerate}
      \item Check that the spin matrices (Equations 4.145 and 4.147) obey the fundamental communication 
      relations for angular momentum, Equation 4.134.

        \textcolor{hwColor}{
          \\
          From the textbook we have;
          \\
          \\
          $
            \begin{cases}
              (4.145): ~~~ S_z=\dfrac{\hbar}{2} \begin{pmatrix}
                1 && 0
                \\
                0 && -1 
              \end{pmatrix}
              \\
              \\
              (4.147): ~~~ S_x=\dfrac{\hbar}{2} \begin{pmatrix}
                0 && 1
                \\
                1 && 0
              \end{pmatrix}, ~~~ S_y=\dfrac{\hbar}{2} \begin{pmatrix}
                0 && -i
                \\
                i && 0
              \end{pmatrix}
            \end{cases}
            \\
            \\
          $
          They should obey the angular momentum, hence:
          \\
          \\
          $
            \begin{cases}
              [S_x, S_y]=i \hbar S_z
              \\
              \\
              [S_y, S_z]=i \hbar S_x
              \\
              \\
              [S_z, S_x]=i \hbar S_y
            \end{cases}
            \\
            \\
            \\
          $
          $
            [S_x, S_y]=S_x S_y-S_y S_x=\dfrac{\hbar^2}{4} \left[
              \begin{pmatrix}
                i && 0
                \\
                0 && -i
              \end{pmatrix}
              -\begin{pmatrix}
                -i && 0
                \\
                0 && i
              \end{pmatrix}
            \right]=2 i \dfrac{\hbar^2}{4} \begin{pmatrix}
              1 && 0
              \\
              0 && -1
            \end{pmatrix}=i \hbar S_z
            \\
            \\
            \\
          $
          $
            [S_y, S_z]=S_y S_z-S_z S_y=\dfrac{\hbar^2}{4} \left[
              \begin{pmatrix}
                0 && i
                \\
                i && 0
              \end{pmatrix}
              -\begin{pmatrix}
                0 && -i
                \\
                -i && 0
              \end{pmatrix}
            \right]=2 i \dfrac{\hbar^2}{4} \begin{pmatrix}
              0 && 1
              \\
              1 && 0
            \end{pmatrix}=i \hbar S_x
            \\
            \\
            \\
          $
          $
            [S_z, S_x]=S_z S_x-S_x S_z=\dfrac{\hbar^2}{4} \left[
              \begin{pmatrix}
                0 && 1
                \\
                -1 && 0
              \end{pmatrix}
              -\begin{pmatrix}
                0 && -1
                \\
                1 && 0
              \end{pmatrix}
            \right]=2 i \dfrac{\hbar^2}{4} \begin{pmatrix}
              0 && -i
              \\
              i && 0
            \end{pmatrix}=i \hbar S_y
            \\
            \\
          $
        }

      \item Show that the Pauli spin matrices (Equation 4.148) satisfy the product rule
      $$
        \sigma_j \sigma_k=\delta_{jk}+i ~ \sum\limits_{l} \epsilon_{jkl} \sigma_l
      $$
      where the indices stand for $x, y,$ or $z$, and $\epsilon_{jkl}$ is the \textbf{Levi-Civita} symbol:
      $+1$ if $jkl=123, 231$ or $312;$ $-1$ if $jkl=132, 213,$ or $321; 0$ otherwise.

        \textcolor{hwColor}{
          \\
          $
            \sigma_x \sigma_y=\delta_{xy}+i ~ \sum\limits_{z} \epsilon_{xyz} \sigma_z
            \\
            \\
            \begin{pmatrix}
              0 && 1
              \\
              1 && 0
            \end{pmatrix} \begin{pmatrix}
              0 && -i
              \\
              i && 0
            \end{pmatrix}=\begin{pmatrix}
              i && 0 
              \\
              0 && -i
            \end{pmatrix}=i \sigma_z
            \\
            \\
            \rule{15cm}{1pt}
            \\
            \\
            \sigma_y \sigma_x=\left[\sigma_x \sigma_y\right]^{\dagger}=-i \sigma_z
            \\
            \\
            \rule{15cm}{1pt}
            \\
            \\
            \sigma_y \sigma_z=\delta_{yz}+i ~ \sum\limits_{x} \epsilon_{yzx} \sigma_x
            \\
            \\
            \begin{pmatrix}
              0 && -i
              \\
              i && 0
            \end{pmatrix} \begin{pmatrix}
              1 && 0
              \\
              0 && -1
            \end{pmatrix}=\begin{pmatrix}
              0 && i 
              \\
              i && 0
            \end{pmatrix}=i \sigma_x
            \\
            \\
            \rule{15cm}{1pt}
            \\
            \\
            \sigma_z \sigma_y=\left[\sigma_y \sigma_z\right]^{\dagger}=-i \sigma_x
            \\
            \\
            \rule{15cm}{1pt}
            \\
            \\
            \sigma_z \sigma_x=\delta_{zx}+i ~ \sum\limits_{y} \epsilon_{zxy} \sigma_y
            \\
            \\
            \begin{pmatrix}
              1 && 0
              \\
              0 && -1
            \end{pmatrix} \begin{pmatrix}
              0 && 1
              \\
              1 && 0
            \end{pmatrix}=\begin{pmatrix}
              0 && 1 
              \\
              -1 && 0
            \end{pmatrix}=i \sigma_y
            \\
            \\
            \rule{15cm}{1pt}
            \\
            \\
            \sigma_x \sigma_z=\left[\sigma_z \sigma_x\right]^{\dagger}=-i \sigma_y
            \\
            \\
            \rule{15cm}{1pt}
            \\
            \\
            \sigma^2_x= \begin{pmatrix}
              0 && 1
              \\
              1 && 0
            \end{pmatrix} \begin{pmatrix}
              0 && 1
              \\
              1 && 0
            \end{pmatrix}=\begin{pmatrix}
              1 && 0 
              \\
              0 && 1
            \end{pmatrix}=\delta_{xx}
            \\
            \\
            \rule{15cm}{1pt}
            \\
            \\
            \sigma^2_y= \begin{pmatrix}
              0 && -i
              \\
              i && 0
            \end{pmatrix} \begin{pmatrix}
              0 && -i
              \\
              i && 0
            \end{pmatrix}=\begin{pmatrix}
              1 && 0 
              \\
              0 && 1
            \end{pmatrix}=\delta_{yy}
            \\
            \\
            \rule{15cm}{1pt}
            \\
            \\
            \sigma^2_z= \begin{pmatrix}
              1 && 0
              \\
              0 && -1
            \end{pmatrix} \begin{pmatrix}
              1 && 0
              \\
              0 && -1
            \end{pmatrix}=\begin{pmatrix}
              1 && 0 
              \\
              0 && 1
            \end{pmatrix}=\delta_{zz}
            \\
            \\
          $
          \\
          \\
          Hence, the product rule is verified.
        }
      
    \end{enumerate}

    \pagebreak

    \item \textbf{4-48} The electron in a hydrogen atom occupies the combined spin and position state
    $$
      \Psi=R_{21} \left(
        \sqrt{\dfrac{1}{3}} Y_1^0 ~ \chi_+ +\sqrt{\dfrac{2}{3}} Y_1^1 \chi_-
      \right)
    $$

    \textcolor{hwColor}{
      \\
      Interpretation:
      \\
      \\
      $
        \begin{cases}
          R_{21} \Longrightarrow n=2, ~ \ell=1
          \\
          \\
          Y_1^0 \Longrightarrow \ell=1, ~ m_{\ell}=0
          \\
          \\
          \chi_+ \Longrightarrow s=\dfrac{1}{2}, ~ m_s=\dfrac{1}{2}
          \\
          \\
          Y_1^1 \Longrightarrow \ell=1, ~ m_{\ell}=1
          \\
          \\
          \chi_- \Longrightarrow s=\dfrac{1}{2}, ~ m_s=-\dfrac{1}{2}
        \end{cases}
        \therefore ~~~ m_{\ell}+m_s=\dfrac{1}{2}
      $
    }

    \begin{enumerate}
      \item If you measured the orbital angular momentum squared ($L^2$), what values might you get,
      and what is the probability of each?

        \textcolor{hwColor}{
          \\
          $
            \ell=1 \Longrightarrow L^2 \Psi=\ell \hbar^2 \left(\ell+1\right) \Psi
            \\
            \\
            \therefore ~~~ \hbar^2 \ell \left(\ell+1\right)=2 \hbar^2 \Longrightarrow P=1
            \\
            \\
            \therefore ~~~~ \text{Probability is $100\%$}. ~~~~ \checkmark
          $
          \\
        }

      \item Same for the z-component of orbital angular momentum ($L_z$).

        \textcolor{hwColor}{
          \\
          We have two possible values for $m_{\ell}$ which are $0$ and $1$.
          \\
          \\
          $
            \begin{cases}
              m_{\ell}=0 \Longrightarrow P=\sqrt{\dfrac{1}{3}}^2=\dfrac{1}{3}
              \\
              \\
              m_{\ell}=1 \Longrightarrow P=\sqrt{\dfrac{2}{3}}^2=\dfrac{2}{3}
            \end{cases} ~~~~~ \checkmark
          $
          \\
        }

      \item Same for the spin angular momentum squared ($S^2$).

        \textcolor{hwColor}{
          \\
          $
            \Psi S^2=\Psi \hbar^2 S(S+1)
            \\
            \\
            S=\dfrac{1}{2} \Longrightarrow \begin{cases}
              P=1
              \\
              \\
              \dfrac{3}{4} \hbar^2
            \end{cases}
          $
          \\
        }

      \item Same for the z-component of spin angular momentum ($S_z$).

        \textcolor{hwColor}{
          \\
          $
            S_z \Psi=\Psi m_s \hbar, m_s=\pm\dfrac{1}{2}
            \\
            \\
            \begin{cases}
              m_s=+\dfrac{1}{2} \Longrightarrow P=\dfrac{1}{3}
              \\
              \\
              m_s=-\dfrac{1}{2} \Longrightarrow P=\dfrac{2}{3}
            \end{cases}
            \\
          $
        }

    \end{enumerate}
  \end{enumerate}

\end{document}