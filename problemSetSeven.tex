\documentclass[fleqn]{article}
\oddsidemargin 0.0in
\textwidth 6.0in
\thispagestyle{empty}
\usepackage{import}
\usepackage{amsmath}
\usepackage{graphicx}
\usepackage{flexisym}
\usepackage{calligra}
\usepackage{amssymb}
\usepackage{bigints} 
\usepackage[english]{babel}
\usepackage[utf8x]{inputenc}
\usepackage{float}
\usepackage[colorinlistoftodos]{todonotes}


\DeclareMathAlphabet{\mathcalligra}{T1}{calligra}{m}{n}
\DeclareFontShape{T1}{calligra}{m}{n}{<->s*[2.2]callig15}{}
\newcommand{\scriptr}{\mathcalligra{r}\,}
\newcommand{\boldscriptr}{\pmb{\mathcalligra{r}}\,}

\definecolor{hwColor}{HTML}{1a0252}

\begin{document}

  \begin{titlepage}

    \newcommand{\HRule}{\rule{\linewidth}{0.5mm}}

    \center

    \begin{center}
      \includegraphics[height=11cm, width=11cm]{asu.png}
    \end{center}

    \vline

    \textsc{\LARGE Quantum Physics II}\\[1.5cm]

    \HRule \\[0.5cm]
    { \huge \bfseries Problem Set Seven}\\[0.4cm] 
    \HRule \\[1.0cm]

    \textbf{Behnam Amiri}

    \bigbreak

    \textbf{Prof: Onur Erten}

    \bigbreak

    \textbf{{\large \today}\\[2cm]}

    \vfill

  \end{titlepage}

  \begin{enumerate}
    \item \textbf{4-24}
    \begin{enumerate}
      \item Check that the spin matrices (Equations 4.145 and 4.147) obey the fundamental commutation relations for angular momentum, Equation 4.134.

        % \textcolor{hwColor}{
        %   \\
        % }

      \item Show that the Pauli spin matrices (Equation 4.148) satisfy the product rule 
      $$
        \sigma_j ~ \sigma_k=\delta_{jk}+i \sum\limits_{\ell} \epsilon_{jk \ell} \sigma_{\ell}
      $$
      where the indices stand for $x, y,$ or $z$, and $\epsilon_{jk \ell}$ is the \textbf{Levi-Civita} symbol: $+1$ if 
      $jk \ell=123, 231,$ or $312; -1$ if $jk \ell=132, 213,$ or $321; 0$ otherwise.

        % \textcolor{hwColor}{
        %   \\
        % }

    \end{enumerate}

    \item \textbf{4-48} The electron in a hydrogen atom occupies the combined spin and position state
    $$
      R_{21} \left(
        \sqrt{\dfrac{1}{3}} Y_1^0 ~ X_+ +\sqrt{\dfrac{2}{3}} Y_1^1 X_-
      \right)
    $$
    \begin{enumerate}
      \item If you measured the orbital angular momentum squared ($L^2$), what values might you get,
      and what is the probability of each?

        % \textcolor{hwColor}{
        %   \\
        % }

      \item Same for the z-component of orbital angular momentum ($L_z$).

        % \textcolor{hwColor}{
        %   \\
        % }

      \item Same for the spin angular momentum squared ($S^2$).

        % \textcolor{hwColor}{
        %   \\
        % }

      \item Same for the z-component of spin angular momentum ($S_z$).

        % \textcolor{hwColor}{
        %   \\
        % }

    \end{enumerate}
    

  \end{enumerate}

\end{document}
