\documentclass[fleqn]{article}
\oddsidemargin 0.0in
\textwidth 6.0in
\thispagestyle{empty}
\usepackage{import}
\usepackage{amsmath}
\usepackage{graphicx}
\usepackage{flexisym}
\usepackage{calligra}
\usepackage{amssymb}
\usepackage{bigints} 
\usepackage[english]{babel}
\usepackage[utf8x]{inputenc}
\usepackage{float}
\usepackage[colorinlistoftodos]{todonotes}


\DeclareMathAlphabet{\mathcalligra}{T1}{calligra}{m}{n}
\DeclareFontShape{T1}{calligra}{m}{n}{<->s*[2.2]callig15}{}
\newcommand{\scriptr}{\mathcalligra{r}\,}
\newcommand{\boldscriptr}{\pmb{\mathcalligra{r}}\,}

\definecolor{hwColor}{HTML}{1a0252}

\begin{document}

  \begin{titlepage}

    \newcommand{\HRule}{\rule{\linewidth}{0.5mm}}

    \center

    \begin{center}
      \includegraphics[height=11cm, width=11cm]{asu.png}
    \end{center}

    \vline

    \textsc{\LARGE Quantum Physics II}\\[1.5cm]

    \HRule \\[0.5cm]
    { \huge \bfseries Problem Set Five}\\[0.4cm] 
    \HRule \\[1.0cm]

    \textbf{Behnam Amiri}

    \bigbreak

    \textbf{Prof: Onur Erten}

    \bigbreak

    \textbf{{\large \today}\\[2cm]}

    \vfill

  \end{titlepage}

  \begin{enumerate}
    \item \textbf{4-15}
    \begin{enumerate}
      \item Find $\langle r \rangle$ and $\langle r^2 \rangle$ for an electron in the ground state of hydrogen. Express your
      answers in terms of the Bohr radius. 


      \item Find $\langle x \rangle$ and $\langle x^2 \rangle$ for an electron in the ground state of hydrogen. 
      \emph{Hint: This requires no new integration-note that $r^2=x^2+y^2+z^2$, and exploit the symmetry of the ground state}



      \item Find $\langle x^2 \rangle$ in the state $n=2$, $\ell$, $m=1$. \emph{Hint: this state is not symmetrical in 
      $x, y,z.$ Use $x=r sin(\theta) cos(\phi)$.}


    \end{enumerate}

    \item \textbf{4-15} What is the \emph{most probable} value of $r$, in the ground state of hydrogen? 
    (The answer is \emph{not zero!}) \emph{Hint:} First you must figure out the probablity that the electron would be found between
    $r$ and $r+dr$.


  \end{enumerate}

\end{document}