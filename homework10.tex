\documentclass[fleqn]{article}
\oddsidemargin 0.0in
\textwidth 6.0in
\thispagestyle{empty}
\usepackage{import}
\usepackage{amsmath}
\usepackage{graphicx}
\usepackage{flexisym}
\usepackage{calligra}
\usepackage{amssymb}
\usepackage{bigints} 
\usepackage[english]{babel}
\usepackage[utf8x]{inputenc}
\usepackage{float}
\usepackage[colorinlistoftodos]{todonotes}


\DeclareMathAlphabet{\mathcalligra}{T1}{calligra}{m}{n}
\DeclareFontShape{T1}{calligra}{m}{n}{<->s*[2.2]callig15}{}
\newcommand{\scriptr}{\mathcalligra{r}\,}
\newcommand{\boldscriptr}{\pmb{\mathcalligra{r}}\,}

\definecolor{hwColor}{HTML}{1a0252}

\begin{document}

  \begin{titlepage}

    \newcommand{\HRule}{\rule{\linewidth}{0.5mm}}

    \center

    \begin{center}
      \includegraphics[height=11cm, width=11cm]{asu.png}
    \end{center}

    \vline

    \textsc{\LARGE Quantum Physics II}\\[1.5cm]

    \HRule \\[0.5cm]
    { \huge \bfseries Problem Set Ten}\\[0.4cm] 
    \HRule \\[1.0cm]

    \textbf{Behnam Amiri}

    \bigbreak

    \textbf{Prof: Onur Erten}

    \bigbreak

    \textbf{{\large \today}\\[2cm]}

    \vfill

  \end{titlepage}

  \begin{enumerate}
    \item Use a gaussian trial function (Equation 8.2) to obtain the lowest upper bound you can on the ground 
    state energy of (a) the linear potential: $V(x)=\alpha |x|;$ (b) the quartic potential: $V(x)=\alpha x^4$.

      \textcolor{hwColor}{
        \\
        From the textbook we have the equation 8.2 as $\psi(x)=A e^{-bx^2}$.
        \\
        \\
        $
          (a)
          \\
          \\
          \langle V \rangle=\bigints\limits_{0}^{\infty} 2 A^2 \alpha x e^{-2bx^2} ~ dx
          =2 A^2 \alpha \big( -\dfrac{e^{-2bx^2}}{4b} \bigg)\Big|_{0}^{\infty}
          \\
          \\
          \\
          \therefore ~~~ \boxed{
            \langle V \rangle=\dfrac{\alpha}{\sqrt{2\pi b}}
          } ~~~~ \checkmark
          \\
          \\
          \\
          \langle H \rangle=\langle V \rangle+\dfrac{\hbar^2 b}{2m}
          \\
          \\
          \\
          \dfrac{\partial \langle H \rangle}{\partial b}=\dfrac{\hbar^2}{2m}-\dfrac{\alpha b^{-3/2}}{\sqrt{8 \pi}}=0
          \Longrightarrow \boxed{
            b=\bigg( \dfrac{m \alpha }{\hbar^2 \sqrt{2 \pi}} \bigg)^{2/3}
          } ~~~~ \checkmark
          \\
          \\
          \\
          \langle H \rangle_{min}=\dfrac{\hbar^2}{2m} \bigg( \dfrac{m \alpha }{\hbar^2 \sqrt{2 \pi}} \bigg)^{2/3}
          +\dfrac{\alpha}{\sqrt{2 \pi}} \bigg( \dfrac{\hbar^2 \sqrt{2 \pi}}{m \alpha} \bigg)^{1/3}
          \\
          \\
          \\
          \therefore ~~~ \boxed{
            \langle H \rangle_{min}=\dfrac{3}{2} \sqrt[3]{\dfrac{\hbar^2 \alpha^2}{2 \pi m}} 
          } ~~~~ \checkmark
          \\
          \\
          \\
          \\
          \\
          \\
          \\
          (b)
          \\
          \\
          \langle V \rangle=\bigints\limits_{0}^{\infty} 2 A^2 \alpha x^4 e^{-2bx^2} ~ dx
          =A^2 \alpha \dfrac{3}{16b^2} \sqrt{\dfrac{\pi}{2b}}
          \\
          \\
          \\
          \therefore ~~~ \boxed{
            \langle V \rangle=\dfrac{3 \alpha}{16 b^2}
          } ~~~~ \checkmark
          \\
          \\
          \\
          \langle H \rangle=\langle V \rangle+\dfrac{\hbar^2 b}{2m}
          \\
          \\
          \\
          \dfrac{\partial \langle H \rangle}{\partial b}=\dfrac{\hbar^2}{2m}-\dfrac{3 \alpha}{8 b^3}=0
          \Longrightarrow b=\sqrt[3]{\dfrac{3 \alpha m}{4 \hbar^2}}
          \\
          \\
          \\
          \langle H \rangle_{min}=\dfrac{\hbar^2}{2m} \sqrt[3]{\dfrac{3 \alpha m}{4 \hbar^2}}
          +\dfrac{3 \alpha}{16} \bigg( \dfrac{4 \hbar^2 }{3 \alpha m} \bigg)^{2/3}
          \\
          \\
          \\
          \therefore ~~~ \boxed{
            \langle H \rangle_{min}=\dfrac{3}{4} \sqrt[3]{\bigg( \dfrac{3 \hbar^4 \alpha}{4 m^2} \bigg)} 
          } ~~~~
          \\
          \\
        $
      }

    \item Estimate the ground state energy of the hydrogen atom by means of the variational method using 
    the following two trial functions, find the relative errors, compare the two results, and discuss the 
    merit of each trial function.
    $$
      (a) ~~~ \phi_a(r)=\begin{cases}
        1-\dfrac{r}{a} r \leq a
        \\
        \\
        0  ~~~~ r > a
      \end{cases}
    $$
    where $a$ is an adjustable parameter. Find a relation between $\alpha_{min}$ and the Bohr radius.
    $$
      (b) ~~~ \phi_a(r)=A ~ e^{-\alpha r^2}
    $$
      \textcolor{hwColor}{
        \\
        $
          (a)
          \\
          \\
          \hat{H}=\dfrac{\overrightarrow{p}^2}{2 \mu }-\dfrac{e^2}{r}
          \\
          \\
          \tilde{E}(\alpha)=\dfrac{
            \langle \phi_{\alpha} |\hat{H}| \phi_{\alpha} \rangle
          }{
            \langle \phi_{\alpha} | \phi_{\alpha} \rangle
          }, ~~~~ \text{knowing that } \dfrac{\partial \tilde{E}}{\partial \alpha}=0
          \\
          \\
          \\
          \langle \phi_{\alpha} | \phi_{\alpha} \rangle
          =\bigints\limits_{0}^{a} ~ \bigints\limits_{0}^{\pi} ~ \bigints\limits_{0}^{2\pi} ~ r^2/2 \bigg( 1-\dfrac{r}{\alpha} \bigg)^2 sin\theta ~ dr ~ d\theta ~ d\phi
          \\
          \\
          \\
          \therefore ~~~ \boxed{
            \langle \phi_{\alpha} | \phi_{\alpha} \rangle=\dfrac{2 \pi \alpha^3}{15}
          } ~~~~
          \\
          \\
          \\
          \\
          \langle \phi_{\alpha} |\hat{H}| \phi_{\alpha} \rangle=
          - \dfrac{4 \pi \hbar^2}{\mu \alpha^2} \bigints\limits_{0}^{a} r^2 dr+\dfrac{4 \pi \hbar^2}{\mu \alpha} \bigints\limits_{0}^{a} rdr-4 \pi e^2 \bigints\limits_{0}^{a} \bigg( r-\dfrac{2r^2}{\alpha}+\dfrac{r^3}{\alpha^2} \bigg)dr
          \\
          \\
          \\
          =-\dfrac{4 \pi \hbar^2 \alpha}{3 \mu}+\dfrac{2 \pi \hbar^2 \alpha}{\mu}-4 \pi e^2 \left[r^2-\dfrac{2r^3}{3 \alpha}+\dfrac{r^4}{4 \alpha^2}\right]\Big|_{0}^{\alpha}
          \\
          \\
          \\
          \langle \phi_{\alpha} |\hat{H}| \phi_{\alpha} \rangle=\dfrac{2 \pi \hbar^2 \alpha}{3 \mu}-\dfrac{7 \pi e^2 \alpha^2}{3}
          \\
          \\
          \\
          -\dfrac{10 \hbar^2}{\mu \alpha^3}+\dfrac{35 e^2}{2 \alpha^2}=0 \Longrightarrow 
          \alpha=\dfrac{4 \hbar^2}{7 e^2 \mu} 
          \\
          \\
          \\
          \text{Estimate of ground state energy eigenvalue: } \tilde{E} \approx -15 \dfrac{\mu e^4}{\hbar^2}.
          \\
          \\
        $
        We know that the energy ground state for a hydrogen atom is $E=-\dfrac{1}{2} \dfrac{\mu e^4}{\hbar^2}$, therefore
        what we found is a lot bigger than the exact solution.
        \\
        \\
        \\
        \\
        \\
        \\
        $
          (b)
          \\
          \\
          \hat{E}(\alpha)=\dfrac{
            \langle \phi_{\alpha} |\hat{H}| \phi_{\alpha} \rangle
          }{
            \langle \phi_{\alpha} | \phi_{\alpha} \rangle
          }
          \\
          \\
          \\
          \langle \phi_{\alpha} | \phi_{\alpha} \rangle
          =\bigints\limits_{0}^{a} \bigints\limits_{0}^{\pi} \bigints\limits_{0}^{2 \pi} A^2 r^2 e^{-2\alpha r^2} sin\theta dr ~ d\theta ~ d\phi
          \\
          \\
          \\
          \therefore ~~~ \boxed{
            \langle \phi_{\alpha} | \phi_{\alpha} \rangle=\dfrac{A^2 \pi}{2\alpha} \sqrt{\dfrac{\pi}{2 \alpha}}, ~~~ A^2=\dfrac{2 \alpha}{\pi} \sqrt{\dfrac{2 \alpha}{\pi}}
          } ~~~~ \checkmark
          \\
          \\
          \\
          \langle \phi_{\alpha} |\hat{H}| \phi_{\alpha} \rangle
          =-\dfrac{4 \pi \hbar^2 A^2}{2 \mu} \bigints\limits_{0}^{\infty} e^{-\alpha r^2} e^{-\alpha r^2} \dfrac{\partial }{\partial r} \bigg( r^2 \dfrac{\partial }{\partial} \bigg) dr
          -4 \pi A^2 e^2 \bigints\limits_{0}^{\infty} r^2 e^{-2\alpha r^2} \dfrac{1}{r} dr
          \\
          \\
          \\
          -\dfrac{\hbar^2 A^2 \pi}{\mu} \bigints\limits_{0}^{\infty} e^{-\alpha r^2} e^{-\alpha r^2} \dfrac{\partial }{\partial r} \bigg( r^2 \dfrac{\partial }{\partial r}\bigg) dr
          =-\dfrac{4 \hbar^2 \alpha}{\mu} \sqrt{\dfrac{2 \alpha}{\pi}} \left[
            -6 \alpha \bigints\limits_{0}^{\infty} r^2 e^{-2 \alpha r^2} dr+\bigints\limits_{0}^{\infty} 4 \alpha^2 r^4 e^{-2 \alpha r^2} dr
          \right]
          =\dfrac{3 \hbar^2 \alpha}{2 \mu}
          \\
          \\
          \\
          \\
          4 \pi A^2 e^2 \bigints\limits_{0}^{\infty} r^2 e^{-2 \alpha r^2} \dfrac{dr}{r}=2 \pi e^2 A^2 \bigints\limits_{0}^{\infty} e^{-2 \alpha r^2} d\xi=2e^2 \sqrt{\dfrac{2 \alpha}{\pi}}, ~~~~ \xi=r^2
          \\
          \\
          \\
          \hat{E}(\alpha)=\dfrac{3 \alpha \hbar^2}{2 \mu}-2e^2 \sqrt{\dfrac{2 \alpha }{\pi}} 
          \Longrightarrow \dfrac{\partial \hat{E}(\alpha)}{\partial \alpha}=\dfrac{3 \hbar^2}{2 \mu}-e^2 \sqrt{\dfrac{2}{\mu \alpha}}=0
          \Longrightarrow \alpha =\dfrac{2}{\pi} \bigg( \dfrac{2 \mu e^2}{3 \hbar^3} \bigg)^2
          \\
          \\
          \\
          \therefore ~~~ \boxed{
            \hat{E}(\alpha)=-\dfrac{4}{3} \dfrac{\mu e^4}{\pi \hbar^2}, ~~~ \dfrac{\hat{E}(\alpha)}{E}=\dfrac{8}{3 \pi}
          } ~~~~ \checkmark
          \\
          \\
        $
      }

  \end{enumerate}

\end{document}
