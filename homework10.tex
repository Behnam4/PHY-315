\documentclass[fleqn]{article}
\oddsidemargin 0.0in
\textwidth 6.0in
\thispagestyle{empty}
\usepackage{import}
\usepackage{amsmath}
\usepackage{graphicx}
\usepackage{flexisym}
\usepackage{calligra}
\usepackage{amssymb}
\usepackage{bigints} 
\usepackage[english]{babel}
\usepackage[utf8x]{inputenc}
\usepackage{float}
\usepackage[colorinlistoftodos]{todonotes}


\DeclareMathAlphabet{\mathcalligra}{T1}{calligra}{m}{n}
\DeclareFontShape{T1}{calligra}{m}{n}{<->s*[2.2]callig15}{}
\newcommand{\scriptr}{\mathcalligra{r}\,}
\newcommand{\boldscriptr}{\pmb{\mathcalligra{r}}\,}

\definecolor{hwColor}{HTML}{1a0252}

\begin{document}

  \begin{titlepage}

    \newcommand{\HRule}{\rule{\linewidth}{0.5mm}}

    \center

    \begin{center}
      \includegraphics[height=11cm, width=11cm]{asu.png}
    \end{center}

    \vline

    \textsc{\LARGE Quantum Physics II}\\[1.5cm]

    \HRule \\[0.5cm]
    { \huge \bfseries Problem Set Ten}\\[0.4cm] 
    \HRule \\[1.0cm]

    \textbf{Behnam Amiri}

    \bigbreak

    \textbf{Prof: Onur Erten}

    \bigbreak

    \textbf{{\large \today}\\[2cm]}

    \vfill

  \end{titlepage}

  \begin{enumerate}
    \item Use a gaussian trial function (Equation 8.2) to obtain the lowest upper bound you can on the ground 
    state energy of (a) the linear potential: $V(x)=\alpha |x|;$ (b) the quartic potential: $V(x)=\alpha x^4$.

      \textcolor{hwColor}{
        \\
        From the textbook we have the equation 8.2 as $\psi(x)=A e^{-bx^2}$.
        \\
        \\
        $
          (a)
          \\
          \\
          \langle V \rangle=\bigints\limits_{0}^{\infty} 2 A^2 \alpha x e^{-2bx^2} ~ dx
          =2 A^2 \alpha \big( -\dfrac{e^{-2bx^2}}{4b} \bigg)\Big|_{0}^{\infty}
          \\
          \\
          \\
          \therefore ~~~ \boxed{
            \langle V \rangle=\dfrac{\alpha}{\sqrt{2\pi b}}
          } ~~~~ \checkmark
          \\
          \\
          \\
          \langle H \rangle=\langle V \rangle+\dfrac{\hbar^2 b}{2m}
          \\
          \\
          \\
          \dfrac{\partial \langle H \rangle}{\partial b}=\dfrac{\hbar^2}{2m}-\dfrac{\alpha b^{-3/2}}{\sqrt{8 \pi}}=0
          \Longrightarrow \boxed{
            b=\bigg( \dfrac{m \alpha }{\hbar^2 \sqrt{2 \pi}} \bigg)^{2/3}
          } ~~~~ \checkmark
          \\
          \\
          \\
          \langle H \rangle_{min}=\dfrac{\hbar^2}{2m} \bigg( \dfrac{m \alpha }{\hbar^2 \sqrt{2 \pi}} \bigg)^{2/3}
          +\dfrac{\alpha}{\sqrt{2 \pi}} \bigg( \dfrac{\hbar^2 \sqrt{2 \pi}}{m \alpha} \bigg)^{1/3}
          \\
          \\
          \\
          \therefore ~~~ \boxed{
            \langle H \rangle_{min}=\dfrac{3}{2} \sqrt[3]{\dfrac{\hbar^2 \alpha^2}{2 \pi m}} 
          } ~~~~ \checkmark
          \\
          \\
          \\
          \\
          \\
          \\
          \\
          (b)
          \\
          \\
          \langle V \rangle=\bigints\limits_{0}^{\infty} 2 A^2 \alpha x^4 e^{-2bx^2} ~ dx
          =A^2 \alpha \dfrac{3}{16b^2} \sqrt{\dfrac{\pi}{2b}}
          \\
          \\
          \\
          \therefore ~~~ \boxed{
            \langle V \rangle=\dfrac{3 \alpha}{16 b^2}
          } ~~~~ \checkmark
          \\
          \\
          \\
          \langle H \rangle=\langle V \rangle+\dfrac{\hbar^2 b}{2m}
          \\
          \\
          \\
          \dfrac{\partial \langle H \rangle}{\partial b}=\dfrac{\hbar^2}{2m}-\dfrac{3 \alpha}{8 b^3}=0
          \Longrightarrow b=\sqrt[3]{\dfrac{3 \alpha m}{4 \hbar^2}}
          \\
          \\
          \\
          \langle H \rangle_{min}=\dfrac{\hbar^2}{2m} \sqrt[3]{\dfrac{3 \alpha m}{4 \hbar^2}}
          +\dfrac{3 \alpha}{16} \bigg( \dfrac{4 \hbar^2 }{3 \alpha m} \bigg)^{2/3}
          \\
          \\
          \\
          \therefore ~~~ \boxed{
            \langle H \rangle_{min}=\dfrac{3}{4} \sqrt[3]{\bigg( \dfrac{3 \hbar^4 \alpha}{4 m^2} \bigg)} 
          } ~~~~
          \\
          \\
        $
      }

    \item Begin by sketching the shapes of the two trial functions, as well as the shape
    of the exact solution. Remember that these functions are $3D$ wave functions in spherical coordinates.
    In order to minimize your chances for mistakes, write out your result for each integral (normalization,
    kinetic \& potential) as you go along.

      % \textcolor{hwColor}{
      %   \\
      % }

  \end{enumerate}

\end{document}
