\documentclass[fleqn]{article}
\oddsidemargin 0.0in
\textwidth 6.0in
\thispagestyle{empty}
\usepackage{import}
\usepackage{amsmath}
\usepackage{graphicx}
\usepackage{flexisym}
\usepackage{calligra}
\usepackage{amssymb}
\usepackage{bigints} 
\usepackage[english]{babel}
\usepackage[utf8x]{inputenc}
\usepackage{float}
\usepackage[colorinlistoftodos]{todonotes}


\DeclareMathAlphabet{\mathcalligra}{T1}{calligra}{m}{n}
\DeclareFontShape{T1}{calligra}{m}{n}{<->s*[2.2]callig15}{}
\newcommand{\scriptr}{\mathcalligra{r}\,}
\newcommand{\boldscriptr}{\pmb{\mathcalligra{r}}\,}

\definecolor{hwColor}{HTML}{1a0252}

\begin{document}

  \begin{titlepage}

    \newcommand{\HRule}{\rule{\linewidth}{0.5mm}}

    \center

    \begin{center}
      \includegraphics[height=11cm, width=11cm]{asu.png}
    \end{center}

    \vline

    \textsc{\LARGE Quantum Physics II}\\[1.5cm]

    \HRule \\[0.5cm]
    { \huge \bfseries Problem Set Eight}\\[0.4cm] 
    \HRule \\[1.0cm]

    \textbf{Behnam Amiri}

    \bigbreak

    \textbf{Prof: Onur Erten}

    \bigbreak

    \textbf{{\large \today}\\[2cm]}

    \vfill

  \end{titlepage}

  \begin{enumerate}
    \item A particle of spin $\dfrac{1}{2}$ is in a orbital $(L=1)$ of some potential.
    \begin{enumerate}
      \item What values of $J$, total angular momentum can it have?

        \textcolor{hwColor}{
          \\
          The total angular momentum vector then is the sum of the total orbital angular momentum vector and 
          the total spin angular momentum vector. 
          \\
          \\
          $
            |\ell-s|=j=\ell+s
            \\
            \\
            j=1-\dfrac{1}{2},1+\dfrac{1}{2}
            \\
            \\
            \\
            \therefore ~~~ j=(\dfrac{1}{2}, \dfrac{3}{2}) ~~~~ \checkmark
            \\
          $
        }

      \item Use the method described in lecture (using the z-components and $\pm$ components of $J=L+S$ to find the state 
      $\vert j=\dfrac{1}{2}, j_z=\dfrac{1}{2} \rangle$ in terms of the basis $\vert l=1,l_z \rangle \vert s=\dfrac{1}{2}, s_z \rangle$.
      (Hint: First workout the $\vert j=\dfrac{3}{2}, j_z=\dfrac{3}{2} \rangle$ then use orthogonality principle).

        \textcolor{hwColor}{
          \\
          $
            J_{\pm}|jm \rangle=\hbar \sqrt{(j \mp m)(j \pm m+1)}|jm \pm 1 \rangle=\hbar \sqrt{j(j+1)-m(m \pm 1)}|jm \pm 1 \rangle
            \\
            \\
            \\
            | \dfrac{3}{2}, \dfrac{3}{2} \rangle=J_- |1,1 \rangle | \dfrac{1}{2}, \dfrac{1}{2} \rangle
            \\
            \\
          $
          R.H.S: $\rightarrow$
          $
            \hbar \sqrt{1(1+1)-1(1-1)} |1 ~ 0 \rangle ~ | \dfrac{1}{2}, \dfrac{1}{2} \rangle
            +\hbar \sqrt{\dfrac{1}{2} (2+1)-\dfrac{1}{2} (\dfrac{1}{2}-1)} |1 ~ 1 \rangle |\dfrac{1}{2}, -\dfrac{1}{2} \rangle
            \\
            \\
            =\hbar \left[
              \sqrt{2} |1 ~ 0\rangle ~ | \dfrac{1}{2}, \dfrac{1}{2} \rangle
              +|1 ~ 1 \rangle |\dfrac{1}{2}, -\dfrac{1}{2} \rangle
            \right]
          $
          \\
          \\
          \\
          \\
          L.H.S: $\rightarrow$
          $
            \hbar \sqrt{\dfrac{3}{2} (\dfrac{3}{2}+1)-\dfrac{3}{2} (\dfrac{3}{2}-1)} |\dfrac{3}{2}, \dfrac{1}{2} \rangle
            =\hbar \sqrt{\dfrac{15}{4}-\dfrac{3}{4}} |\dfrac{3}{2}, \dfrac{1}{2} \rangle
            =\hbar \sqrt{3} |\dfrac{3}{2}, \dfrac{1}{2} \rangle
          $
          \\
          \\
          \\
          By equating the L.H.S and R.H.S we get:
          \\
          \\
          $
            |\dfrac{3}{2} ~ \dfrac{1}{2}=\dfrac{\sqrt{2}}{\sqrt{3}} |1 ~ 0 \rangle |\dfrac{1}{2} ~ \dfrac{1}{2} \rangle
            +\dfrac{1}{\sqrt{3}} |1 ~ 1 \rangle |\dfrac{1}{2} ~ -\dfrac{1}{2} \rangle
          $
          \\
          \\
          \\
          \\
          This state is orthogonal to $|\dfrac{1}{2}, \dfrac{1}{2} \rangle$. 
          \\
          \\
          \\
          $
            |\dfrac{1}{2}, \dfrac{1}{2} \rangle=X|1 ~ 0 \rangle ~ |\dfrac{1}{2} ~ \dfrac{1}{2} \rangle+Y|1 ~ 1 \rangle ~ |\dfrac{1}{2} ~ -\dfrac{1}{2}
            \\
            \\
            \\
            \text{Since} ~~~ \langle \dfrac{1}{2} ~ \dfrac{1}{2} | \dfrac{3}{2} ~ \dfrac{1}{2} \rangle=0 \Longrightarrow \begin{cases}
              X=\dfrac{1}{\sqrt{3}}
              \\
              \\
              Y=-\sqrt{\dfrac{2}{3}}
            \end{cases} ~~~ \checkmark
          $
          \\
          \\
        }

      \item If the system is initially prepared in the state $\vert j=\dfrac{1}{2}, j_z=\dfrac{1}{2} \rangle$, what is the probability that
      measuring its value of $s_z$ would give $\dfrac{1}{2}$?

        \textcolor{hwColor}{
          \\
          we are given the state initial state as $\vert \dfrac{1}{2}, \dfrac{1}{2} \rangle$. From the result of previous section we have:
          \\
          \\
          $
            | \dfrac{1}{2} ~ \dfrac{1}{2}=\dfrac{1}{\sqrt{3}} |1 ~ 0 \rangle | \dfrac{1}{2} ~ \dfrac{1}{2} \rangle-\dfrac{\sqrt{3}}{2} | 1 ~ 1 \rangle | \dfrac{1}{2} ~ -\dfrac{1}{2} \rangle
            \\
            \\
            \{ \langle l ~ l_z | \langle \dfrac{1}{2} ~ \dfrac{1}{2} | \} s_z |\dfrac{1}{2} ~ \dfrac{1}{2} \rangle
            \\
            \\
            \\
            \langle l ~ l_z | \langle \dfrac{1}{2} ~ \dfrac{1}{2} | ~ s_z \left(
              \dfrac{1}{\sqrt{3}} |1 ~ 0 \rangle |\dfrac{1}{2} ~ \dfrac{1}{2} \rangle-\sqrt{\dfrac{2}{3}} |1 ~ 1 \rangle | \dfrac{1}{2} ~ -\dfrac{1}{2} \rangle
            \right)=\dfrac{1}{\sqrt{3}} \dfrac{\hbar}{2}
            \\
            \\
            \\
          $
          Hence, the probability is $\dfrac{1}{3}. ~~~~ \checkmark$
          \\
          \\ 
        }

    \end{enumerate}

    \item \textbf{Zettili 8-4}
    Two identical particles of spin $\dfrac{1}{2}$ are enclosed in a one-dimensional box potential of length $L$ with rigid walls at 
    $x=0$ and $x=L$. Assuming that the two-particle system is in a triplet spin state, find the energy levels, the wave functions, and the degeneracies corresponding
    to the three lowest states.

      \textcolor{hwColor}{
        \\
        A particle in quantum mechanics has only a certain number of labels you can assign to it, namely its quantum numbers: 
        including mass, spin, charge, etc. In quantum mechanics, identical particles (also called indistinguishable or indiscernible 
        particles) are particles that cannot be distinguished from one another, even in principle.
        \\
        \\
        On page $469$ of the textbook we have that there are three states (a triplet) that are symmetric, $\chi_{s}\left(\overrightarrow{S_1}, \overrightarrow{S_2}\right)$
        $$
          \chi_{triplet}\left(\overrightarrow{S_1}, \overrightarrow{S_2}\right)
          =\begin{cases}
            \vert \dfrac{1}{2} ~~ \dfrac{1}{2} \rangle_1 ~~  \vert \dfrac{1}{2} ~~ \dfrac{1}{2} \rangle_2,
            \\
            \\
            \dfrac{1}{\sqrt{2}} \left(
              \vert \dfrac{1}{2} ~~ \dfrac{1}{2} \rangle_1 ~~ \vert \dfrac{1}{2} ~~ -\dfrac{1}{2} \rangle_2
              +\vert \dfrac{1}{2} ~~ -\dfrac{1}{2} \rangle_1 ~~ \vert \dfrac{1}{2} ~~ \dfrac{1}{2} \rangle_2
            \right),
            \\
            \\
            \vert \dfrac{1}{2} ~~ -\dfrac{1}{2} \rangle_1 ~~  \vert \dfrac{1}{2} ~~ -\dfrac{1}{2} \rangle_2,
          \end{cases}
        $$
        From equation $8.62$ we know that $\Psi=\psi(r_1, r_2) ~ \chi(S_1, S_2)$. By the help of equation $8.65$ we have
        \\
        \\
        $
          \psi=\dfrac{1}{\sqrt{2}} \left[
            \psi_{n1}(r_1) ~ \psi_{n2}(r_2)-\psi_{n1}(r_2) ~ \psi_{n2}(r_1)
          \right]
          \\
          \\
          =\dfrac{\sqrt{2}}{L} \left[
            sin(n_1 \pi r_1/L) ~ sin(n_2 \pi r_2/L)
            -sin(n_2 \pi r_1/L) ~ sin(n_1 \pi r_2/L)
          \right]
        $
        \\
        \\
        We got the energy level for the two particle as $E=\dfrac{\hbar^2 \pi^2}{2m L^2} \left(n^2_1+n^2_2\right)$. Let's
        find the energy for the ground state.
        \\
        \\
        $
          E=\dfrac{\hbar^2 \pi^2}{2m L^2} \left(1^2+2^2\right)
          \\
          \\
          \\
          \therefore ~~~ E=\dfrac{5 \hbar^2 \pi^2}{2m L^2}  ~~~~ \checkmark
          \\
          \\
          \\
          \therefore ~~~ \Psi=
          \dfrac{\sqrt{2}}{L} \left[
            sin(n_1 \pi r_1/L) ~ sin(n_2 \pi r_2/L)
            -sin(n_2 \pi r_1/L) ~ sin(n_1 \pi r_2/L)
          \right]
           ~ \chi(S_1, S_2)
        $
        \\
        \\
        For the first excited state we got $\begin{cases}
          n_1=1
          \\
          \\
          n_2=3
        \end{cases} \Longrightarrow E=\dfrac{\hbar^2 \pi^2}{2m L^2} \left(1^2+3^2\right)=\dfrac{5 \hbar^2 \pi^2}{m L^2}
        $
        \\
        \\
        \\
        \\
        For the second excited state we got $\begin{cases}
          n_1=2
          \\
          \\
          n_2=3
        \end{cases} \Longrightarrow E=\dfrac{\hbar^2 \pi^2}{2m L^2} \left(2^2+3^2\right)=\dfrac{13 \hbar^2 \pi^2}{2m L^2}
        $
        \\
        \\
        \\
        Time to plug these results into equation $8.62$.
        \\
        \\
        $
          \Psi=\psi(r_1, r_2) ~ \chi(S_1, S_2)
          \\
          \\
          \\
          \therefore ~~~ \Psi=\dfrac{\sqrt{2}}{L} \left(
            sin(\dfrac{2\pi r_1}{L}) ~ sin(\dfrac{3 \pi r_2}{L})
            -sin(\dfrac{3 \pi r_1}{L}) sin(\dfrac{\pi r_2}{L})
          \right) ~ \chi(S_1, S_2) ~~~~ \checkmark
        $
        \\
        \\
        \\
        Lastly, we can conclude that the energy levels are 3-fold degenerate. (There are three different spin)
        \\
      }
  
  \end{enumerate}

\end{document}
